\documentclass[letterpaper]{recipe}
\usepackage[T1]{fontenc}
\usepackage[italian]{babel}
\usepackage{times,enumerate,ifthen}
\usepackage[usenames]{color}

\newcommand{\bsi}[2]{%
  \fontencoding{T1}\fontfamily{pbs}\fontseries{xl}\fontshape{n}%
  \fontsize{#1}{#2}\selectfont}

\renewcommand{\inghead}{\textbf{Ingredients}:\ }
%\renewcommand{\rechead}{\centering\bsi{24pt}{30pt}}
\renewcommand{\rechead}{\centering\huge\sffamily\bfseries\color{red}}
\makeatletter \setlength{\@totalleftmargin}{0pt}
\renewcommand*\l@subsubsection{\@dottedtocline{3}{3em}{0em}}
\makeatother \setlength\parindent{0pt} \setlength\parskip{2ex plus 0.5ex}
 \pagestyle{empty}
\begin{document}
\recipe{ Matt's Candied Bacon} \ingred{1 pound bacon, 1 cup brown sugar, 1 teaspoon
ground cayenne pepper (optional)}


\begin{enumerate}[{\color{blue} 1.}]
\addtolength{\itemindent}{2em}
\item Preheat oven to 325 degrees.
\item Mix brown sugar and cayenne pepper in a loaf pan or bowl.
\item Coat each strip of bacon with sugar/pepper mixture.  There should be a thin, even coat of sugar
along the entire strip, with no big chunks.
\item Lay strips next to each other (do not stack them) in a large cookie sheet with aluminum foil.
\item Place cookie sheet in oven for {\bf 30 minutes}.
\item Carefully remove bacon from cookie sheet using a fork.  Place bacon on a plate.
\item Use a paper towel to dab off pools of bacon grease.
\item Allow bacon grease to cool before throwing away the aluminum foil, grease, and burnt sugar.
\end{enumerate}
\end{document}
