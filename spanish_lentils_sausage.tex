\documentclass[letterpaper]{recipe}
\usepackage[T1]{fontenc}
\usepackage[italian]{babel}
\usepackage{times,enumerate,ifthen}
\usepackage[usenames]{color}

\newcommand{\bsi}[2]{%
  \fontencoding{T1}\fontfamily{pbs}\fontseries{xl}\fontshape{n}%
  \fontsize{#1}{#2}\selectfont}

\renewcommand{\inghead}{\textbf{Ingredients}:\ }
%\renewcommand{\rechead}{\centering\bsi{24pt}{30pt}}
\renewcommand{\rechead}{\centering\huge\sffamily\bfseries\color{red}}
\makeatletter \setlength{\@totalleftmargin}{0pt}
\renewcommand*\l@subsubsection{\@dottedtocline{3}{3em}{0em}}
\makeatother \setlength\parindent{0pt} \setlength\parskip{2ex plus 0.5ex}
 \pagestyle{empty}
\begin{document}
\recipe{ Spanish-Style Braised Lentils with Sausage }
\ingred{ 1 teaspoon vegetable oil, 1 pound kielbasa sausage cut into 4 equal length and split lengthwise, 1 minced onion,
salt, pepper, 2 minced garlic cloves, 1 teaspoon sweet paprika, 1 teaspoon chili powder, 1 teaspoon minced fresh or 1/4 teaspoon dry thyme,
1 14.5-ounce can diced tomatoes, drained, 1 1/2 cups low-sodium chicken broth, 1 1/2 cups water, 1 cup du Puy, brown, or green lentils, picked over and rinsed,
2 scallions, sliced thin }

\begin{enumerate}[{\color{blue} 1.}]
\addtolength{\itemindent}{2em}
\item Heat the oil in a 12-inch skillet over medium-high heat until shimmering.  Lay the sausage in the skillet cut-side down.  Cook until browned, about 2 minutes.  Transfer to a plate, leaving the fat in the pan.
\item Return the skillet with the fat to medium-high heat until shimmering.  Add the onion and 1/2 teaspoon salt and cook until softened, about 5 minutes.
\item Stir in the garlic, paprika, chili powder, and thyme.  Cook until fragrant, about 15 seconds.
\item Stir in the tomatoes, broth, water, and lentils.  Bring to a boil.  Cover, reduce heat to low, and cook until the lentils are tender but still a little crunchy, about 35 minutes.
\item Nestle the kielbasa into the lentils and continue to cook, covered, over low heat until the lentils are completely tender, about 10 minutes.
\item Season with salt and pepper to taste.  Sprinkle with scallions before serving.
\end{enumerate}
\end{document}
